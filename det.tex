%% LyX 2.0.5.1 created this file.  For more info, see http://www.lyx.org/.
%% Do not edit unless you really know what you are doing.
\documentclass[english]{amsart}
\usepackage[T1]{fontenc}
\usepackage[latin9]{inputenc}
\usepackage{color}
\usepackage{verbatim}
\usepackage{prettyref}
\usepackage{float}
\usepackage{amsthm}
\usepackage{amssymb}
\usepackage[authoryear]{natbib}

\makeatletter

%%%%%%%%%%%%%%%%%%%%%%%%%%%%%% LyX specific LaTeX commands.
\floatstyle{ruled}
\newfloat{algorithm}{tbp}{loa}
\providecommand{\algorithmname}{Algorithm}
\floatname{algorithm}{\protect\algorithmname}

%%%%%%%%%%%%%%%%%%%%%%%%%%%%%% Textclass specific LaTeX commands.
 \usepackage{algorithmic}
\theoremstyle{plain}
\newtheorem{thm}{\protect\theoremname}
  \theoremstyle{plain}
  \newtheorem{lem}[thm]{\protect\lemmaname}
  \theoremstyle{definition}
  \newtheorem{defn}[thm]{\protect\definitionname}
  \theoremstyle{definition}
  \newtheorem*{example*}{\protect\examplename}
  \theoremstyle{remark}
  \newtheorem{rem}[thm]{\protect\remarkname}
  \theoremstyle{definition}
  \newtheorem{example}[thm]{\protect\examplename}

%%%%%%%%%%%%%%%%%%%%%%%%%%%%%% User specified LaTeX commands.


\usepackage{fullpage}\usepackage{amsthm}\usepackage{stmaryrd}

\@ifundefined{definecolor}
 {\usepackage{color}}{}
\usepackage{refstyle}\usepackage{float}\usepackage{amsthm}

\usepackage{a4wide}\ifx\pdfpageheight\undefined
   \usepackage[dvips,colorlinks=true,linkcolor=blue,citecolor=red,
      urlcolor=green]{hyperref}\makeatletter
   \edef\Gin@extensions{\Gin@extensions,.mps}
   
   \DeclareGraphicsRule{.mps}{eps}{*}{}
   \makeatother
\else
   \usepackage[bookmarksopen=false,pdftex=true,breaklinks=true,
       hyperindex=true,pdfstartview=FitH,colorlinks=true,
      pdfpagelabels=true,colorlinks=true,linkcolor=blue,
      citecolor=red,urlcolor=green,hypertexnames=false
      ]{hyperref}\fi

%%%%%%%%%%%%%%%%%%%%%%%%%%%%%% Textclass specific LaTeX commands.
\usepackage{algorithm}\usepackage{algorithmic}\floatstyle{ruled}
\newfloat{algorithm}{tbp}{loa}
\floatname{algorithm}{Algorithm}
\usepackage{algorithmic}\newcommand{\forbody}[1]{ #1 \ENDFOR }
\newcommand{\ifbody}[1]{ #1  \ENDIF}
%\newcommand{\comment}[1]{#1}
\newcommand{\whilebody}[1]{ #1  \ENDWHILE}
\renewcommand{\algorithmicprint}{\textbf{draw}}
\renewcommand{\algorithmicrequire}{\textbf{Input:}}
\renewcommand{\algorithmicensure}{\textbf{Output:}}


%\newtheorem{thm}{Theorem}
%\newtheorem{cor}{Corollary}
%\newtheorem{prop}{Proposition}
%\newtheorem{defn}{Definition}
%\newtheorem{Algo}{Algorithm}
\newtheorem{exmp}{Example}
%\newtheorem{defn}[thm]{Definition}
%\newtheorem{rem}{Remark}
%\newtheorem{lem}{Lemma}


\newenvironment{Alg}{\noindent  {\bf Algorithm} \hspace*{0.05cm}}{

}\newcommand{\Z}{\mathbb{Z}}
\newcommand{\Q}{\mathbb{Q}}
\newcommand{\N}{\mathbb{N}}
\newcommand{\K}{ {\rm K}}
\newcommand{\revCol}{ {\rm revCol}}
\newcommand{\bigO}{\mathcal{O}}
\newcommand{\tbigO}{\widetilde{\mathcal{O}}}
\newcommand{\cL}{\mathcal{L}}
\newcommand{\ocL}{\overline{\mathcal{L}}}
\newcommand{\tcL}{\widetilde{\mathcal{L}}}
\newcommand{\cA}{\mathcal{A}}
\newcommand{\oA}{\overline{A}}
\newcommand{\GL}{{\rm GL}\,}
\newcommand{\rank}{{\rm rank}\,}
\newcommand{\cdeg}{{\rm cdeg}\,}
\newcommand{\rdeg}{{\rm rdeg}\,}
\newcommand{\diag}{{\rm diag}\,}
\newcommand{\val}{{\rm val}\,}
\newcommand{\ord}{{\rm ord}\,}
\newcommand{\abs}[1]{\lvert#1\rvert}

\newcommand{\Return}{\textbf{Return}}
\newcommand{\While}{\textbf{While}}
\newcommand{\For}{\textbf{For}}
\newcommand{\If}{\textbf{If}}
\newcommand{\nv}{\textbf{MatrixPolynomialInverse}}
\newcommand{\then}{\textbf{then}}
\newcommand{\ddo}{\textbf{do}}
\newcommand{\edo}{\textbf{end do}}
\newcommand{\eif}{\textbf{end if}}
\newcommand{\cdim}{{\rm coldim}}
\newcommand{\tO}{O^{\sim}}

\def\StorjohannTransform{\qopname\relax n{StorjohannTransform}}
\def\TransformUnbalanced{\qopname\relax n{TransformUnbalanced}}
\def\rowDimension{\qopname\relax n{rowDimension}}
\def\columnDimension{\qopname\relax n{columnDimension}}
\DeclareMathOperator{\re}{rem}
\DeclareMathOperator{\coeff}{coeff}
\DeclareMathOperator{\lcoeff}{lcoeff}
\DeclareMathOperator{\inv}{inverse}
\DeclareMathOperator{\rev}{rev}
\DeclareMathOperator{\colRev}{colRev}
\DeclareMathOperator{\rowRev}{rowRev}
\DeclareMathOperator{\unimodularCompletion}{unimodularCompletion}
\DeclareMathOperator{\hermiteDiagonal}{HermiteDiagonal}
\DeclareMathOperator{\determinant}{determinant}
\DeclareMathOperator{\mnbr}{MinimalKernelBasisTranspose}
\DeclareMathOperator{\mnbrp}{MinimalKernelBasisWithRankProfile}
\DeclareMathOperator{\colBasis}{ColumnBasis}
\DeclareMathOperator{\rankProfile}{rankProfile}
\def\mab{\qopname\relax n{OrderBasis}}
\def\mnbrs{\qopname\relax n{MinimalNullspacerBasisRankSensitive}}
\def\mmab{\qopname\relax n{FastBasis}}
\def\umab{\qopname\relax n{UnbalancedFastOrderBasis}}
\def\mnb{\qopname\relax n{MinimalKernelBasis ~ }}
\newcommand{\bb}{\\}

\makeatother

\usepackage{babel}
  \providecommand{\definitionname}{Definition}
  \providecommand{\examplename}{Example}
  \providecommand{\lemmaname}{Lemma}
  \providecommand{\remarkname}{Remark}
\providecommand{\theoremname}{Theorem}

\begin{document}

\title{Computing Determinant of Polynomial Matrices}


\author{Wei Zhou and George Labahn}


\thanks{Cheriton School of Computer Science, University of Waterloo, Waterloo
ON, Canada N2L 3G1 \textbf{\{w2zhou,glabahn\}@uwaterloo.ca}}
\begin{abstract}
Given a square, nonsingular matrix of univariate polynomials $\mathbf{F}\in\mathbb{K}[x]^{n\times n}$
over a field $\mathbb{K}$, we give a fast, deterministic algorithm
for finding the determinant of $\mathbf{F}$ with complexity $O^{\sim}\left(n^{\omega}s\right)$
where $s$ is the average column degree or the average row degree
of $\mathbf{F}$. Here soft-$O$ notation is Big-$O$ with log factors
removed and $\omega$ is the exponent of matrix multiplication. 
\end{abstract}
\maketitle

\section{Introduction}

For a given square, nonsingular matrix polynomial $\mathbf{F}\in\mathbb{K}[x]^{n\times n}$
there exists a unimodular matrix $\mathbf{U}$ such that $\mathbf{F}\cdot\mathbf{U}=\mathbf{H}$,
a matrix in (column) Hermite normal form. Thus 
\[
\mathbf{H}=\left[\begin{array}{cccc}
h_{11}\\
h_{21} & h_{22}\\
\vdots & \vdots & \ddots\\
h_{n1} & \cdots & \cdots & h_{nn}
\end{array}\right]
\]
 a lower triangular matrix where $h_{ii}$ is monic and deg $h_{ij}<\mbox{ deg }h_{ii}$
for all $j<i$. Other variations include specifying row rather than
column forms (in which case the unimodular matrix multiplies on the
left rather than the right) or upper rather than lower triangular
form. The Hermite form was first defined by Hermite in 1851 in the
context of triangularizing integer matrices.

There has been considerable work on fast algorithms for Hermite form
computation. This includes $O^{\sim}\left(n^{4}d\right)$ algorithms
from Hafner and McCurley \citet{hafner} and Iliopoulos \citet{iliopoulos}
which control intermediate size by working modulo the determinant.
Hafner and McCurley \citet{hafner}, Storjohann and Labahn \citet{storjohann-labahn96}
and Villard \citet{villard96} gave new algorithms which reduced the
cost to $O^{\sim}\left(n^{\omega+1}d\right)$ operations, with $2<\omega<3$
being the exponent of matrix multiplication. The second named worked
with integer matrices but the results carried over directly to polynomial
matrices. Mulders and Storjohann \citet{mulders-storjohann:2003}
gave an iterative algorithm having complexity $O\left(n^{3}d^{2}\right)$,
thus reducing the exponent of $n$ at the cost of increasing the exponent
of the degree $d$.

During the past decade the goal has been to give an algorithm that
computes the Hermite form in the time it takes to multiply two polynomial
matrices having the same size $n$ and degree $d$ as the input matrix,
namely at a cost $O^{\sim}\left(n^{\omega}d\right)$. Such algorithms
already exist for a number of other polynomial matrix problems. This
included probabalistic algorithms for example linear solving \citet{mulders-storjohann:2003},
row reduction \citet{Giorgi2003} and polynomial matrix inversion
\citet{jeannerod-villard:05} and later deterministic algorithms for
linear solving and row reduction \citet{GSSV2012}. In the case of
Hermite normal form computation Gupta and Storjohann \citet{GS2011}
gave an algorithm which reduced the complexity to $O^{\sim}\left(n^{3}d\right)$.
Their algorithm is randomized of Las Vegas type rather than deterministic.

In this paper we give a deterministic Hermite norml form algorithm
having complexity $O^{\sim}\left(n^{3}d\right)$. As in Gupta and
Storjohann our approach is given in two steps : first find the diagonal
elements of the Hermite form and then find the remaining entries.
While Gupta and Storjohann use the Smith normal form to determine
the diagonal entries our approach is to make use of fast, deterministic
methods forshifted minimal kernel basis and column basis computation.
The use of shifted minimal kernel bases for matrix normal form computation
was previously used in \citet{BLV:1999,BLV:jsc06} in order to obtain
efficient algorithms in the case where intermediate coefficient growth
is a concern.

Our algorithm finds the diagonal entries of the Hermite form has a
cost of $O^{\sim}\left(n^{\omega-1}s\right)$ field operations where
$s$ is the average of the column degrees of $\mathbf{F}$. By slightly
modifying the diagonal algorithm we are also able to give as a corollary
a $O^{\sim}\left(n^{\omega-1}s\right)$ deterministic algorithm for
finding the determinant of $\mathbf{F}$. This compares to the $O^{\sim}\left(n^{\omega}d\right)$
cost for the Las Vegas determinant algorithm of Storjohann given in
\citet{storjohann:2002,storjohann:2003}

The remainder of this paper is organized as follows. In the next section
we give preliminary information for shifted degrees, kernel and column
bases of polynomial matrices. Section 3 then contains the algorithm
for finding the diagonal elements of a Hermite form with the following
section giving the details of the fast algorithm for the entire Hermite
normal form computation. Section 5 describes the small modification
needed for determinant computation. The paper ends with a conclusion
and topics for future research. 



\section{Preliminaries}

In this section we first describe the notations used in this paper,
and then give the basic definitions and properties of {\em shifted
degree}, {\em kernel basis} and {\em column basis} for a matrix
of polynomials. These will be the building blocks used in our algorithm.


\subsection{Shifted Degrees}

Our methods makes use of the concept of {\em shifted} degrees of
polynomial matrices \citep{BLV:1999}, basically shifting the importance
of the degrees in some of the rows of a basis. For a column vector
$\mathbf{p}=\left[p_{1},\dots,p_{n}\right]^{T}$ of univariate polynomials
over a field $\mathbb{K}$, its column degree, denoted by $\cdeg\mathbf{p}$,
is the maximum of the degrees of the entries of $\mathbf{p}$, that
is, 
\[
\cdeg~\mathbf{p}=\max_{1\le i\le n}\deg p_{i}.
\]
 The \emph{shifted column degree} generalizes this standard column
degree by taking the maximum after shifting the degrees by a given
integer vector that is known as a \emph{shift}. More specifically,
the shifted column degree of $\mathbf{p}$ with respect to a shift
$\vec{s}=\left[s_{1},\dots,s_{n}\right]\in\mathbb{Z}^{n}$, or the
\emph{$\vec{s}$-column degree} of $\mathbf{p}$ is 
\[
\cdeg_{\vec{s}}~\mathbf{p}=\max_{1\le i\le n}[\deg p_{i}+s_{i}]=\deg(x^{\vec{s}}\cdot\mathbf{p}),
\]
 where 
\[
x^{\vec{s}}=\diag\left(x^{s_{1}},x^{s_{2}},\dots,x^{s_{n}}\right)~.
\]
 For a matrix $\mathbf{P}$, we use $\cdeg\mathbf{P}$ and $\cdeg_{\vec{s}}\mathbf{P}$
to denote respectively the list of its column degrees and the list
of its shifted $\vec{s}$-column degrees. When $\vec{s}=\left[0,\dots,0\right]$,
the shifted column degree specializes to the standard column degree.
Similarly, $\cdeg_{-\vec{s}}\mathbf{P}\leq0$ is equivalent to deg
$p_{ij}\leq s_{i}$ for all $i$ and $j$, that is, $\vec{s}$ bounds
the row degrees of $\mathbf{P}$.

The shifted row degree of a row vector \textbf{$\mathbf{q}=\left[q_{1},\dots,q_{n}\right]$}
is defined similarly as 
\[
\rdeg_{\vec{s}}\mathbf{q}=\max_{1\le i\le n}[\deg q_{i}+s_{i}]=\deg(\mathbf{q}\cdot x^{\vec{s}}).
\]
 Shifted degrees have been used previously in polynomial matrix computations
and in generalizations of some matrix normal forms \citep{BLV:jsc06}.
The shifted column degree is equivalent to the notion of \emph{defect}
commonly used in the literature.

Along with shifted degrees we also make use of the notion of a matrix
polynomial being column (or row) reduced. A matrix polynomial $\mathbf{F}$
is column reduced if the leading column coefficient matrix, that is
the matrix 
\[
[\mbox{coeff}(f_{ij},x,d_{j})]_{1\leq i,j\leq n},\mbox{ with }\vec{d}=\mbox{cdeg }\mathbf{F},
\]
 has full rank. A matrix polynomial $\mathbf{F}$ is $\vec{s}$ column
reduced if $x^{\vec{s}}\mathbf{F}$ is column reduced. A similar concept
exists for being shifted row reduced.

The usefulness of the shifted degrees can be seen from their applications
in polynomial matrix computation problems \citep{ZL2012,za2012}.
One of its uses is illustrated by the following lemma from \citep[Chapter 2]{zhou:phd2012},
which can be viewed as a stronger version of the predictable-degree
property \citep[page 387]{kailath:1980}. For completeness we also
include the proof. 
\begin{lem}
\label{lem:predictableDegree} Let $\mathbf{A}\in\mathbb{K}\left[x\right]^{m\times n}$
be a $\vec{u}$-column reduced matrix with no zero columns and with
$\cdeg_{\vec{u}}\mathbf{A}=\vec{v}$. Then a matrix $\mathbf{B}\in\mathbb{K}\left[x\right]^{n\times k}$
has $\vec{v}$-column degrees $\cdeg_{\vec{v}}\mathbf{B}=\vec{w}$
if and only if $\cdeg_{\vec{u}}\left(\mathbf{A}\mathbf{B}\right)=\vec{w}$. \end{lem}
\begin{proof}
Being $\vec{u}$-column reduced with $\cdeg_{\vec{u}}\mathbf{A}=\vec{v}$
is equivalent to the leading coefficient matrix of $x^{\vec{u}}\cdot\mathbf{A}\cdot x^{-\vec{v}}$
having linearly independent columns. The leading coefficient matrix
of $x^{\vec{v}}\cdot\mathbf{B}\cdot x^{-\vec{w}}$ has no zero column
if and only if the leading coefficient matrix of 
\[
x^{\vec{u}}\cdot\mathbf{AB}\cdot x^{-\vec{w}}=x^{\vec{u}}\cdot\mathbf{A}\cdot x^{-\vec{v}}x^{\vec{v}}\cdot\mathbf{B}\cdot x^{-\vec{w}}.
\]
 That is, $x^{\vec{v}}\cdot\mathbf{B}\cdot x^{-\vec{w}}$ has column
degree $\vec{0}$ if and only if $x^{\vec{u}}\cdot\mathbf{AB}\cdot x^{-\vec{w}}$
has column degree $\vec{0}$. 
\end{proof}
An essential fact needed in this paper, also based on the use of shifted
degrees, is the efficient multiplication of matrices with unbalanced
degrees \citep[Theorem 3.7]{za2012}. 
\begin{thm}
\label{thm:multiplyUnbalancedMatrices} Let $\mathbf{A}\in\mathbb{K}\left[x\right]^{m\times n}$
with $m\le n$, $\vec{s}\in\mathbb{Z}^{n}$ a shift with entries bounding
the column degrees of $\mathbf{A}$ and $\xi$, a bound on the sum
of the entries of $\vec{s}$. Let $\mathbf{B}\in\mathbb{K}\left[x\right]^{n\times k}$
with $k\in O\left(m\right)$ and the sum $\theta$ of its $\vec{s}$-column
degrees satisfying $\theta\in O\left(\xi\right)$. Then we can multiply
$\mathbf{A}$ and $\mathbf{B}$ with a cost of $O^{\sim}(n^{2}m^{\omega-2}s)\subset O^{\sim}(n^{\omega}s)$
field operations, where $s=\xi/n$ is the average of the entries of
$\vec{s}$. 
\end{thm}

\subsection{Kernel Bases}

The kernel of $\mathbf{F}\in\mathbb{K}\left[x\right]^{m\times n}$
is the $\mathbb{F}\left[x\right]$-module 
\[
\left\{ \mathbf{p}\in\mathbb{K}\left[x\right]^{n}~|~\mathbf{F}\mathbf{p}=0\right\} 
\]
 with a kernel basis of $\mathbf{F}$ being a basis of this module.
Formally, we have:
\begin{defn}
\label{def:kernelBasis}Given $\mathbf{F}\in\mathbb{K}\left[x\right]^{m\times n}$,
a polynomial matrix $\mathbf{N}\in\mathbb{K}\left[x\right]^{n\times k}$
is a (right) kernel basis of $\mathbf{F}$ if the following properties
hold: 
\begin{enumerate}
\item $\mathbf{N}$ is full-rank. 
\item $\mathbf{N}$ satisfies $\mathbf{F}\cdot\mathbf{N}=0$. 
\item Any $\mathbf{q}\in\mathbb{K}\left[x\right]^{n}$ satisfying $\mathbf{F}\mathbf{q}=0$
can be expressed as a linear combination of the columns of $\mathbf{N}$,
that is, there exists some polynomial vector $\mathbf{p}$ such that
$\mathbf{q}=\mathbf{N}\mathbf{p}$. 
\end{enumerate}
\end{defn}
It is not difficult to show that any pair of kernel bases $\mathbf{N}$
and $\mathbf{M}$ of $\mathbf{F}$ %are column bases of each other and 
are unimodularly equivalent.

A $\vec{s}$-minimal kernel basis of $\mathbf{F}$ is just a kernel
basis that is $\vec{s}$-column reduced. 
\begin{defn}
Given $\mathbf{F}\in\mathbb{K}\left[x\right]^{m\times n}$, a polynomial
matrix $\mathbf{N}\in\mathbb{K}\left[x\right]^{n\times k}$ is a $\vec{s}$-minimal
(right) kernel basis of $\mathbf{F}$ if\textbf{ $\mathbf{N}$} is
a kernel basis of $\mathbf{F}$ and $\mathbf{N}$ is $\vec{s}$-column
reduced. We also call a $\vec{s}$-minimal (right) kernel basis of
$\mathbf{F}$ a $\left(\mathbf{F},\vec{s}\right)$-kernel basis.

%We will need to use the following bound on the sizes of kernel bases
%from \cite{za2012}.

\end{defn}
%\begin{thm}
%\label{thm:boundOfSumOfShiftedDegreesOfKernelBasis}Suppose %$\mathbf{F}\in\mathbb{K}\left[x\right]^{m\times n}$
%and $\vec{s}\in\mathbb{Z}_{\ge0}^{n}$ is a shift with entries bounding
%the corresponding column degrees of $\mathbf{F}$. Then the sum of
%the $\vec{s}$-column degrees of any $\vec{s}$-minimal kernel basis
%of $\mathbf{F}$ is bounded by $\sum\vec{s}$.
%\end{thm}
%We will need the following result from \cite{za2012} to compute kernel bases by rows. 
%\begin{thm}
%\label{thm:continueComputingKernelBasisByRows}Let $\mathbf{G}=\left[\mathbf{G}_{1}^{T},%\mathbf{G}_{2}^{T}\right]^{T}\in\mathbb{K}\left[x\right]^{m\times n}$
%and $\vec{t}\in\mathbb{Z}^{n}$ a shift vector. If $\mathbf{N}_{1}$
%is a $\left(\mathbf{G}_{1},\vec{t}\right)$-kernel basis with $\vec{t}$-column
%degrees $\vec{u}$, and $\mathbf{N}_{2}$ is a $\left(\mathbf{G}_{2}\mathbf{N}_{1},\vec{u}\right)$-%kernel
%basis with $\vec{u}$-column degrees $\vec{v}$, then $\mathbf{N}_{1}\mathbf{N}_{2}$
%is a $\left(\mathbf{G},\vec{t}\right)$-kernel basis $\vec{t}$-column
%degrees $\vec{v}$. 
%\end{thm}


%\subsection{Column Bases}


We will need the following result from \citep{za2012} to bound the
sizes of kernel bases.
\begin{thm}
\label{thm:boundOfSumOfShiftedDegreesOfKernelBasis}Suppose $\mathbf{F}\in\mathbb{K}\left[x\right]^{m\times n}$
and $\vec{s}\in\mathbb{Z}_{\ge0}^{n}$ is a shift with entries bounding
the corresponding column degrees of $\mathbf{F}$. Then the sum of
the $\vec{s}$-column degrees of any $\vec{s}$-minimal kernel basis
of $\mathbf{F}$ is bounded by $\sum\vec{s}$. 
\end{thm}
A column basis of $\mathbf{F}$ is a basis for the $\mathbb{K}\left[x\right]$-module
\[
\left\{ \mathbf{F}\mathbf{p}~|~\mathbf{p}\in\mathbb{K}\left[x\right]^{n}~\right\} ~.
\]
 Such a basis can be represented as a full rank matrix $\mathbf{T}\in\mathbb{K}\left[x\right]^{m\times r}$
whose columns are the basis elements. A column basis is not unique
and indeed any column basis right multiplied by a unimodular polynomial
matrix gives another column basis.

The cost of kernel basis computation is given in \citep{za2012} while
the cost of column basis computation is given in \citep{za2013}.
In both cases they make heavy use of fast methods for order bases
(often also referred to as minimal approximant bases) \citep{BeLa94,Giorgi2003,ZL2012}.
\begin{thm}
\label{thm:costGeneral} Let $\mathbf{F}\in\mathbb{K}\left[x\right]^{m\times n}$
with $\vec{s}=\cdeg\mathbf{F}$. Then a $\left(\mathbf{F},\vec{s}\right)$-kernel
basis can be computed with a cost of $O^{\sim}\left(n^{\omega}s\right)$field
operations where $s=\sum\vec{s}/n$ is the average column degree of
$\mathbf{F}$. 
\end{thm}
~
\begin{thm}
\label{thm:fastcolbasis} There exists a fast, deterministic algorithm
for the computation of a column basis of a matrix polynomial $\mathbf{F}$
having complexity $O^{\sim}\left(nm^{\omega-1}s\right)$ field operations
in $\mathbb{K}$ with $s$ being the average average column degree
of $\mathbf{F}$. In addition, the column basis computed has column
degrees bounded by the $r$ largest column degrees of $\mathbf{F}$,
where $r$ is the rank of $\mathbf{F}$.
\end{thm}
Column bases and kernel bases are closely related, as shown by the
following result from \citet{za2013,zhou:phd2012}.
\begin{lem}
\label{lem:unimodular_kernel_columnBasis} Let $\mathbf{F}\in\mathbb{K}\left[x\right]^{m\times n}$
and suppose $\mathbf{U}\in\mathbb{K}\left[x\right]^{n\times n}$ is
a unimodular matrix such that $\mathbf{F}\mathbf{U}=\left[0,\mathbf{T}\right]$
with $\mathbf{T}$ of full column rank. Partition $\mathbf{U}=\left[\mathbf{U}_{L},\mathbf{U}_{R}\right]$
such that $\mathbf{F}\cdot\mathbf{U}_{L}=0$ and $\mathbf{F}\mathbf{U}_{R}=\mathbf{T}$.
Then 
\begin{enumerate}
\item $\mathbf{U}_{L}$ is a kernel basis of $\mathbf{F}$ and $\mathbf{T}$
is a column basis of~$\mathbf{F}$. 
\item If $\mathbf{N}$ is any other kernel basis of $\mathbf{F}$, then
$\mathbf{U}^{*}=\left[\mathbf{N},~\mathbf{U}_{R}\right]$ is also
unimodular and also unimodularly transforms $\mathbf{F}$ to $\left[0,\mathbf{T}\right]$. 
\end{enumerate}
\end{lem}
~
\begin{lem}
\label{lem:colBasisdegreeBoundByRdegOfRightFactor}A column basis
$\mathbf{T}$ of $\mathbf{F}$ computed using the algorithm 

and $\mathbf{G}$ be as before and $\vec{t}=-\rdeg_{-\vec{s}}\mathbf{G}$.
Then 
\begin{itemize}
\item [(i)] the column degrees of $\mathbf{T}$ are bounded by the corresponding
entries of $\vec{t}$; 
\item [(ii)] if $\vec{t}$ has $r$ entries and $\vec{s}^{~\prime}$ is
the list of the $r$ largest entries of $\vec{s}$, then $\vec{t}\le\vec{s}^{~\prime}$. 
\end{itemize}
\end{lem}

\subsection{Example}
\begin{example}
\label{ex:example1} Let 
\[
\mathbf{F}=\left[\begin{array}{rcrcr}
x & -{x}^{3} & -2\,{x}^{4} & 2x & -{x}^{2}\\
\noalign{\medskip}1 & -1 & -2\, x & 2 & -x\\
\noalign{\medskip}-3 & 3\,{x}^{2}+x & 2\,{x}^{2} & -\,{x}^{4}+1 & 3\, x
\end{array}\right]
\]
 be a $3\times5$ matrix over $\mathbb{Z}_{7}[x]$ having column degree
$\vec{s}=(1,3,4,4,2)$. Then a column space, $\mathbf{G}$, and a
kernel basis, $\mathbf{N}$, of $\mathbf{F}$ are given by 
\[
\mathbf{G}=\left[\begin{array}{rcr}
x & -{x}^{3} & -2\,{x}^{4}\\
\noalign{\medskip}1 & -1 & -2\, x\\
\noalign{\medskip}-3 & 3\,{x}^{2}+x & 2\,{x}^{2}
\end{array}\right]~~\mbox{ and }~~\mathbf{N}:=\left[\begin{array}{rc}
-1 & x\\
\noalign{\medskip}-{x}^{2} & 0\\
\noalign{\medskip}-3\, x & 0\\
\noalign{\medskip}-3 & 0\\
\noalign{\medskip}0 & 1
\end{array}\right]~.
\]
 For example, if $\{\mathbf{g}_{i}\}_{i=1,...,5}$ denote the columns
of $\mathbf{G}$ then column $4$ of $\mathbf{F}$ - denoted by $\mathbf{f}_{4}$
- is given by 
\[
\mathbf{f}_{4}=-2~\mathbf{g}_{1}-2x^{2}~\mathbf{g}_{2}+x~\mathbf{g}_{3}+2~\mathbf{g}_{4}.
\]
 Here $\cdeg_{\vec{s}}\mathbf{N}=(5,2)$ with shifted leading coefficient
matrix 
\[
\mbox{lcoeff}_{\vec{s}}(\mathbf{N})=\left[\begin{array}{rc}
0 & 1\\
-1 & 0\\
-3 & 0\\
0 & 0\\
0 & 1
\end{array}\right].
\]
 Since $\mbox{ lcoeff}_{\vec{s}}(\mathbf{N})$ has full rank we have
that $\mathbf{N}$ is a $\vec{s}$-minimal kernel basis. \qed \end{example}




\section{\label{sec:diagonals}Recursive Computation}

In this section we show how to recursively compute the determinant
of a nonsingular input matrix $\mathbf{F}\in\mathbb{K}\left[x\right]^{n\times n}$
with column degrees $\vec{s}$. The computation makes use of fast
kernel and column basis computation.

Consider unimodularly transforming $\mathbf{F}$ to 
\begin{equation}
\mathbf{F}\mathbf{U}=\mathbf{G}=\begin{bmatrix}\mathbf{G}_{1} & 0\\
* & \mathbf{G}_{2}
\end{bmatrix},\label{eq:step1HermiteDiagonal}
\end{equation}
which eliminates a top right block and gives two square diagonal blocks
$\mathbf{G}_{1}$ and\textbf{ $\mathbf{G}_{2}$} in $\mathbf{G}$.
Then the determinant of $\mathbf{F}$ can be computed as 
\begin{equation}
\det\mathbf{F}=\frac{\det\mathbf{G}}{\det\mathbf{U}}=\frac{\det\mathbf{G}_{1}\cdot\det\mathbf{G}_{2}}{\det\mathbf{U}},\label{eq:determinantFromDiagonalBlocks}
\end{equation}
 which requires us to first compute $\det\mathbf{G}_{1}$, $\det\mathbf{G}_{2}$,
and $\det\mathbf{U}$. The same procedure can then be applied recursively
to compute the determinant of $\mathbf{G}_{1}$ and the determinant
of $\mathbf{G}_{2}$. This can be repeated recursively until the dimension
becomes 1. 

One major obstacle of this approach, however, is that the degrees
of the unimodular matrix $\mathbf{U}$ and the matrix $\mathbf{G}$
can be too large for efficient computation. %
\begin{comment}
To construct such an example, just use a random $k\times\left(k+1\right)$
matrix $\mathbf{A}$, and set $\mathbf{F}=\begin{bmatrix}\mathbf{A} & 0\\
* & *
\end{bmatrix}$, and compute $\mathbf{G}_{1}$ as a column basis of $\mathbf{A}$.
Then the degrees of $\mathbf{U}$ and $\mathbf{G}$ are often $\Theta(nd)$.
\end{comment}
{} To sidestep this, we will show that the matrices $\mathbf{G}_{1}$,\textbf{
$\mathbf{G}_{2}$}, and $\det\mathbf{U}$ can be in fact computed
without computing the matrices $\mathbf{G}$ and $\mathbf{U}$. 


\subsection{Computing the diagonal blocks}

Suppose we want $\mathbf{G}_{1}$ to have dimension $k$. We can partition\textbf{
$\mathbf{F}$ }as $\mathbf{F}=\begin{bmatrix}\mathbf{F}_{U}\\
\mathbf{F}_{D}
\end{bmatrix}$ with $k$ rows in $\mathbf{F}_{U}$, then both $\mathbf{F}_{U}$
and $\mathbf{F}_{D}$ are of full-rank since $\mathbf{F}$ is assumed
to be nonsingular. By partitioning $\mathbf{U}=\begin{bmatrix}\mathbf{U}_{L}~~, & \mathbf{U}_{R}\end{bmatrix}$,
with $k$ columns in $\mathbf{U}_{L}$, then 
\begin{equation}
\mathbf{F}\mathbf{U}=\begin{bmatrix}\mathbf{F}_{U}\\
\mathbf{F}_{D}
\end{bmatrix}\begin{bmatrix}\mathbf{U}_{L} & \mathbf{U}_{R}\end{bmatrix}=\begin{bmatrix}\mathbf{G}_{1} & 0\\
* & \mathbf{G}_{2}
\end{bmatrix}=\mathbf{G}.\label{eq:UPartitioned}
\end{equation}
 Notice that the matrix $\mathbf{G}_{1}$ is nonsingular and is therefore
a column basis of $\mathbf{F}_{U}$. As such this can be efficiently
computed %using \cite{za2013} 
as mentioned in Theorem \ref{thm:fastcolbasis}. In addition, the
column basis algorithm makes the resulting column degrees of $\mathbf{G}_{1}$
small enough for $\mathbf{G}_{1}$ to be efficiently used again as
the input matrix of a new subproblem in the recursive procedure.
\begin{lem}
\label{lem:firstDiagonalBlock}The first diagonal block $\mathbf{G}_{1}$
in $\mathbf{G}$ can be computed with a cost of $O^{\sim}\left(n^{\omega}s\right)$
and with column degrees bounded by the $k$ largest column degrees
of $\mathbf{F}$.
\end{lem}
For computing the second diagonal block $\mathbf{G}_{2}$, notice
that we do not need a complete unimodular matrix $\mathbf{U}$, only
$\mathbf{U}_{R}$ is needed to compute $\mathbf{G}_{2}=\mathbf{F}_{D}\mathbf{U}_{R}$.
In fact, \prettyref{lem:unimodular_kernel_columnBasis} tells us much
more. It tells us that the matrix $\mathbf{U}_{R}$ is a right kernel
basis of $\mathbf{F}$, which makes the top right block of $\mathbf{G}$
zero, and the kernel basis $\mathbf{U}_{R}$ can be replaced by any
other kernel basis of $\mathbf{F}$ to give another unimodular matrix
that also transforms $\mathbf{F}_{U}$ to a column basis and also
eliminates the top right block of $\mathbf{G}$. 
\begin{lem}
\label{lem:oneStepHermiteDiagonal} Partition $\mathbf{F}=\begin{bmatrix}\mathbf{F}_{U}\\
\mathbf{F}_{D}
\end{bmatrix}$ and suppose $\mathbf{G}_{1}$ is a column basis of $\mathbf{F}_{U}$
and $\mathbf{N}$ a kernel basis of $\mathbf{F}_{U}$. Then there
is a unimodular matrix $\mathbf{U}=\left[~*~,~\mathbf{N}\right]$
such that 
\[
\mathbf{F}\mathbf{U}=\begin{bmatrix}\mathbf{G}_{1} & 0\\
* & \mathbf{G}_{2}
\end{bmatrix},
\]
 where $\mathbf{G}_{2}=\mathbf{F}_{D}\mathbf{N}$. If $\mathbf{F}$
is square nonsingular, then $\mathbf{G}_{1}$ and $\mathbf{G}_{2}$
are also square nonsingular. 
\end{lem}
Note that the block represented by the symbol $*$ is not needed in
our computation. This block may have very large degrees and cannot
be computed efficiently. Now we can also determine $\mathbf{G}_{1}$
and $\mathbf{G}_{2}$ without computing the unimodular matrix $\mathbf{U}$.
%$\mathbf{G}_{1}$ can be computed using the method from \cite{za2013},
%while the kernel basis computation from \cite{za2012}
%can be used to compute a kernel basis $\mathbf{N}$ of $\mathbf{F}_{U}$,
%which can then in turn be used to compute $\mathbf{G}_{2}=\mathbf{F}_{D}\mathbf{N}$.
%The particular kernel basis $\mathbf{N}$ found from \cite{za2012} has the degree bounds 
%needed in Theorem \ref{thm:multiplyUnbalancedMatrices} and hence this multiplication can also be done efficiently.
%After $\mathbf{G}_{1}$ and $\mathbf{G}_{2}$ are computed, we can
%repeat the same process on each of these two matrices, which now have
%lower dimensions, until the dimension becomes one. 
 

For this recursive procedure to be efficient, we must be able to compute
$\mathbf{G}_{2}$ efficiently, which can be done by using the existing
algorithms for kernel basis computation and the multiplication of
matrices with unbalanced degrees. We also require that the column
degrees of $\mathbf{G}_{2}$ to be small enough for $\mathbf{G}_{2}$
to be efficiently used again as the input matrix of a new subproblem
in the recursive procedure.
\begin{lem}
\label{lem:secondDiagonalBlock}We can compute the second diagonal
block $\mathbf{G}_{2}$ with a cost of $O^{\sim}\left(n^{\omega}s\right)$
field operations and have $\sum\cdeg\mathbf{G}_{2}\le\sum\vec{s}$.\end{lem}
\begin{proof}
We know $\mathbf{G}_{2}=\mathbf{F}_{D}\mathbf{N}$ from \prettyref{lem:oneStepHermiteDiagonal}
for a kernel basis $\mathbf{N}$ of $\mathbf{F}_{U}$. In fact, this
kernel basis can be made $\vec{s}$-minimal using the algorithm from
\citet{za2012}, and computing such a $\vec{s}$-minimal kernel basis
of $\mathbf{F}_{U}$ costs $O^{\sim}\left(n^{\omega}s\right)$ field
operations by \prettyref{thm:costGeneral}. In addition, the sum of
the $\vec{s}$-column degrees of such a $\vec{s}$-minimal $\mathbf{N}$
is bounded by $\sum\vec{s}$.

For the matrix multiplication $\mathbf{F}_{D}\mathbf{N}$, the sum
of the column degrees of $\mathbf{F}_{D}$ and the sum of the $\vec{s}$-column
degrees of $\mathbf{N}$ are both bounded by $\sum\vec{s}$. Therefore
Theorem \ref{thm:multiplyUnbalancedMatrices} (see also \citet[Theorem 3.7]{za2012})
applies and the multiplication can be done with a cost of $O^{\sim}\left(n^{\omega}s\right)$.

With $\mathbf{N}$ computed, we can then apply \prettyref{thm:multiplyUnbalancedMatrices}
directly to multiply $\mathbf{F}_{D}$ and $\mathbf{N}$ with a cost
of $O^{\sim}\left(n^{\omega}s\right)$ field operations. 
\end{proof}
%Let us look at the computational cost of Algorithm \prettyref{alg:hermiteDiagonal}.
%and \prettyref{alg:hermiteDiagonalWithScalingFactor}. 




\subsection{Determinant of the unimodular matrix}

\prettyref{lem:firstDiagonalBlock} and \prettyref{lem:secondDiagonalBlock}
show that the two diagonal blocks in \prettyref{eq:step1HermiteDiagonal}
can be computed efficiently. To compute the determinant of $\mathbf{F}$
based on \prettyref{eq:determinantFromDiagonalBlocks}, we still need
to know the determinant of the unimodular matrix $\mathbf{U}$ that
satisfies \prettyref{eq:UPartitioned}, or equivalently, we can also
find out the determinant of $\mathbf{V}=\mathbf{U}^{-1}$. Note the
computation of the diagonal blocks in $\mathbf{G}$ also gives $\mathbf{U}_{R}$,
the right $n-k$ columns of $\mathbf{U}$, which is also a right kernel
basis of $\mathbf{F}_{U}$. In fact, it also gives us the matrix $\mathbf{V}_{U}$
consisting of the top $k$ rows of $\mathbf{V}$, from computing $\mathbf{G}_{1}$,
a column basis of $\mathbf{F}_{U}$.
\begin{lem}
Let $k$ be the dimension of $\mathbf{G}_{1}$. The matrix $\mathbf{V}_{U}\in\mathbb{K}\left[x\right]^{k\times n}$
satisfies $\mathbf{G}_{1}\mathbf{V}_{U}=\mathbf{F}_{U}$ if and only
if $\mathbf{V}_{U}$ is the submatrix of $\mathbf{V}=\mathbf{U}^{-1}$
consisting of the top $k$ rows of $\mathbf{V}$. \end{lem}
\begin{proof}
This can be seen from 
\[
\mathbf{G}\mathbf{V}=\begin{bmatrix}\mathbf{G}_{1} & 0\\
* & \mathbf{G}_{2}
\end{bmatrix}\begin{bmatrix}\mathbf{V}_{U}\\
\mathbf{V}_{D}
\end{bmatrix}=\begin{bmatrix}\mathbf{F}_{U}\\
\mathbf{F}_{D}
\end{bmatrix}=\mathbf{F}.
\]

\end{proof}
While the determinant of $\mathbf{V}$ or the determinant of $\mathbf{U}$
is needed to compute the determinant of $\mathbf{F}$, a major problem
is that we do not know $\mathbf{U}_{L}$ or $\mathbf{V}_{D}$, which
may not be efficiently computed due to their large degrees. This means
we need to compute the determinant of $\mathbf{V}$ or $\mathbf{U}$
without knowing the complete matrix $\mathbf{V}$ or $\mathbf{U}$.
The following lemma shows how this can be done using just $\mathbf{U}_{R}$
and $\mathbf{V}_{U}$.
\begin{lem}
\label{lem:scalingToDeterminant} Let $\mathbf{U}=\left[\mathbf{U}_{L},\mathbf{U}_{R}\right]$
and $\mathbf{F}$ be as before, that is, they satisfy 
\[
\mathbf{F}\mathbf{U}=\begin{bmatrix}\mathbf{F}_{U}\\
\mathbf{F}_{D}
\end{bmatrix}\begin{bmatrix}\mathbf{U}_{L} & \mathbf{U}_{R}\end{bmatrix}=\begin{bmatrix}\mathbf{G}_{1} & 0\\
* & \mathbf{G}_{2}
\end{bmatrix}=\mathbf{G},
\]
 where the row dimension of $\mathbf{F}_{U}$, the column dimension
of $\mathbf{U}_{L}$, and the dimension of $\mathbf{G}_{1}$ are $k$.
Let $\mathbf{V}=\begin{bmatrix}\mathbf{V}_{U}\\
\mathbf{V}_{D}
\end{bmatrix}$ be the inverse of $\mathbf{U}$ with $k$ rows in $\mathbf{V}_{U}$.
If $\mathbf{U}_{L}^{*}\in\mathbb{K}\left[x\right]^{n\times k}$ is
a matrix such that $\mathbf{U}^{*}=\left[\mathbf{U}_{L}^{*},\mathbf{U}_{R}\right]$
is unimodular, then 
\[
\det\mathbf{F}~=~\frac{\det\mathbf{G}\cdot\det\left(\mathbf{V}_{U}\mathbf{U}_{L}^{*}\right)}{\det\left(\mathbf{U}^{*}\right)}.
\]
\end{lem}
\begin{proof}
Since $\det\mathbf{F}~=~\det\mathbf{G}\cdot\det\mathbf{V}$, we just
need to show that $\det\mathbf{V}=\det\left(\mathbf{V}_{U}\mathbf{U}_{L}^{*}\right)/\det\left(\mathbf{U}^{*}\right)$,
which follows from 
\begin{eqnarray*}
\det\mathbf{V}\cdot\det\mathbf{U}^{*} & = & \det\left(\mathbf{V}\cdot\mathbf{U}^{*}\right)\\
 & = & \det\left(\begin{bmatrix}\mathbf{V}_{U}\\
\mathbf{V}_{D}
\end{bmatrix}\begin{bmatrix}\mathbf{U}_{L}^{*} & \mathbf{U}_{R}\end{bmatrix}\right)\\
 & = & \det\left(\begin{bmatrix}\mathbf{V}_{U}\mathbf{U}_{L}^{*} & 0\\
* & I
\end{bmatrix}\right)\\
 & = & \det\left(\mathbf{V}_{U}\mathbf{U}_{L}^{*}\right).
\end{eqnarray*}

\end{proof}
\prettyref{lem:scalingToDeterminant} shows that the determinant of
$\mathbf{V}$ can be computed using $\mathbf{V}_{U}$ and $\mathbf{U}_{R}$,
which we already know from the computation of the diagonal blocks
in $\mathbf{G}$, and a unimodular completion of $\mathbf{U}_{R}$.
In fact, this can be made more efficient still by noticing that the
higher degree parts do not affect the computation.
\begin{lem}
\label{lem:determinantOfUnimodular}If $\mathbf{U}\in\mathbb{K}\left[x\right]^{n\times n}$
is unimodular, then $\det\mathbf{U}=\det\left(\mathbf{U}\mod x\right)=\det\left(\mathbf{U}\left(0\right)\right)$.\end{lem}
\begin{proof}
Note that $\det\left(\mathbf{U}\left(\alpha\right)\right)=\left(\det\mathbf{U}\right)\left(\alpha\right)$
for any $\alpha\in\mathbb{K}$, that is, the result is the same whether
we do evaluation before or after computing the determinant. Taking
$\alpha=0$, then we have 
\[
\det\left(\mathbf{U}\mod x\right)=\det\left(\mathbf{U}\left(0\right)\right)=\left(\det\mathbf{U}\right)\left(0\right)=\det\left(\mathbf{U}\right)\mod x=\det\mathbf{U}.
\]

\end{proof}
This allows us to use just the degree zero coefficient matrices in
the computation. So \prettyref{lem:scalingToDeterminant} can be improved
as follows.
\begin{lem}
\label{lem:scalingToDeterminantSimplified} Let $\mathbf{F}$, $\mathbf{U}=\left[\mathbf{U}_{L},\mathbf{U}_{R}\right]$,
and $\mathbf{V}=\begin{bmatrix}\mathbf{V}_{U}\\
\mathbf{V}_{D}
\end{bmatrix}$ be as before. Let $U_{R}=\mathbf{U}_{R}\mod x$, $V_{U}=\mathbf{V}_{U}\mod x$
be the constant matrices of $\mathbf{U}_{R}$ and $\mathbf{V}_{U}$,
respectively. If $U_{L}^{*}\in\mathbb{K}^{n\times*}$ is a matrix
such that $U^{*}=\left[U_{L}^{*},U_{R}\right]$ is unimodular, then
\[
\det\mathbf{F}~=~\frac{\det\mathbf{G}\cdot\det\left(V_{U}U_{L}^{*}\right)}{\det\left(U^{*}\right)}.
\]
 \end{lem}
\begin{proof}
\prettyref{lem:determinantOfUnimodular} implies that $\det\mathbf{V}=\det V$
and $\det\mathbf{U}^{*}=\det U^{*}$, which can be then substituted
in the proof of \prettyref{lem:scalingToDeterminant} to obtain the
result.\end{proof}
\begin{rem}
\prettyref{lem:scalingToDeterminantSimplified} requires us to compute
$U_{L}^{*}\in\mathbb{K}^{n\times*}$ a matrix such that $U^{*}=\left[U_{L}^{*},U_{R}\right]$
is unimodular. This can be obtained easily from the unimodular matrix
that transforms $V_{U}$ to its reduced column echelon form computed
using the Gauss Jordan transform algorithm from \citep{storjohann:phd2000}
with a cost of $O\left(nm^{\omega-1}\right)$. 
\end{rem}
We now have all the ingredients needed for computing the determinant
of $\mathbf{F}$. A recursive algorithm is given in \prettyref{alg:determinant},
which computes the determinant of $\mathbf{F}$ as the product of
the determinant of $\mathbf{V}$ and the determinant of $\mathbf{G}$.
The determinant of $\mathbf{G}$ is computed by recursively computing
the determinants of its diagonal blocks $\mathbf{G}_{1}$ and $\mathbf{G}_{2}$.

\begin{algorithm}[t]
\caption{$\determinant(\mathbf{F})$}
\label{alg:determinant}

\begin{algorithmic}[1]
\REQUIRE{$\mathbf{F}\in\mathbb{K}\left[x\right]^{n\times n}$ is nonsingular.
}

\ENSURE{the determinant of $\mathbf{F}$.}



\STATE{\textbf{if }$n=1$ \textbf{then} \textbf{return} $\mathbf{F}$; \textbf{endif};}

\STATE{$\begin{bmatrix}\mathbf{F}_{U}\\
\mathbf{F}_{D}
\end{bmatrix}:=\mathbf{F}$, with $\mathbf{F}_{U}$ consists of the top $\left\lceil n/2\right\rceil $
rows of $\mathbf{F}$;}

\STATE{$\mathbf{G}_{1},\mathbf{U}_{R},\mathbf{V}_{U}:=\colBasis(\mathbf{F}_{U})$;
(Here we make $\colBasis()$ also return the kernel basis and the
right factor it computed.)}





\STATE{$\mathbf{G}_{2}:=\mathbf{F}_{D}\mathbf{U}_{R}$;}

\STATE{$U_{R}:=\mathbf{U}_{R}\mod x$; $V_{U}:=\mathbf{V}_{U}\mod x$;}

\STATE{compute $U_{L}^{*}\in\mathbb{K}^{n\times k}$ , a matrix that makes
the matrix $U^{*}=\left[U_{L}^{*},U_{R}\right]$ unimodular;}

\STATE{$d_{V}:=\det\left(V_{U}U_{L}^{*}\right)/\det(U^{*};$}

\STATE{$\mathbf{d}_{G}:=\determinant(\mathbf{G}_{1})\cdot\determinant(\mathbf{G}_{2});$}

\STATE{\textbf{return} $d_{V}\cdot\mathbf{d}_{1}\cdot\mathbf{d}_{2}$;}
\end{algorithmic}
\end{algorithm}


\begin{example}
Let 
\[
\mathbf{F}=\left[\begin{array}{rcccc}
x & -{x}^{3} & -2\,{x}^{4} & 2x & -{x}^{2}\\
\noalign{\medskip}1 & -1 & -2\, x & 2 & -x\\
\noalign{\medskip}-3 & 3\,{x}^{2}+x & 2\,{x}^{2} & -\,{x}^{4}+1 & 3\, x\\
\noalign{\medskip}0 & 1 & {x}^{2}+2\, x-2 & \,{x}^{3}+2x-2 & 0\\
\noalign{\medskip}1 & -{x}^{2}+2 & -2\,{x}^{3}-3\, x+3 & 2x+2 & 0
\end{array}\right]
\]
 working over $\mathbb{Z}_{7}[x]$. If $\mathbf{F}_{U}$ denotes the
top three rows of $\mathbf{F}$ then a column basis 
\[
\mathbf{G}_{1}=\left[\begin{array}{rcr}
x & -{x}^{3} & -2\,{x}^{4}\\
\noalign{\medskip}1 & -1 & -2\, x\\
\noalign{\medskip}-3 & 3\,{x}^{2}+x & 2\,{x}^{2}
\end{array}\right]
\]
 and kernel basis 
\[
\mathbf{U}_{R}=\left[\begin{array}{rc}
-1 & x\\
-x^{2} & 0\\
-3x & 0\\
-3 & 0\\
0 & 1
\end{array}\right]
\]
 were given in Example \ref{ex:example1}. The computation of the
column basis also gives the right factor 
\[
\mathbf{V}_{U}=\left[\begin{array}{cccrr}
1 & 0 & 0 & 2 & -x\\
0 & 1 & 0 & 2x^{2} & 0\\
0 & 0 & 1 & -x & 0
\end{array}\right].
\]
Then 
\[
{U}_{R}=\left[\begin{array}{rc}
-1 & 0\\
0 & 0\\
0 & 0\\
-3 & 0\\
0 & 1
\end{array}\right]~\mbox{and}~~{V}_{U}=\left[\begin{array}{cccrr}
1 & 0 & 0 & 2 & 0\\
0 & 1 & 0 & 0 & 0\\
0 & 0 & 1 & 0 & 0
\end{array}\right].
\]
A unimodular completion of $U_{R}$ is given by 
\[
U_{L}^{*}=\left[\begin{array}{ccc}
1 & 0 & 0\\
0 & 1 & 0\\
0 & 0 & 1\\
0 & 0 & 0\\
0 & 0 & 0
\end{array}\right].
\]
 The determinant of $\mathbf{V}$ can then be computed as 
\[
\frac{\det\left(V_{U}U_{L}^{*}\right)}{\det\left(U^{*}\right)}=-\frac{1}{3}=2.
\]
From recursively computing the determinant of $\mathbf{G}_{1}$ and
the determinant of $\mathbf{G}_{2}$, we obtain $\det\mathbf{G}_{1}=x^{6}-x^{4}$
and $\det\mathbf{G}_{2}=x^{4}-x$. 

Then the determinant of $\mathbf{V}$ can be computed as
\[
\det\mathbf{V}=\det\mathbf{V}\cdot\det\mathbf{G}_{1}\cdot\det\mathbf{G}_{2}=2\left(x^{6}-x^{4}\right)\left(x^{4}-x\right)=2x^{10}-2x^{8}-2x^{7}+2x^{5}.
\]
\qed \end{example}




\subsection{Computational cost}
\begin{thm}
\label{thm:diagonalCost} Algorithm \prettyref{alg:determinant} %and \prettyref{alg:hermiteDiagonalWithScalingFactor}
costs $O^{\sim}\left(n^{\omega}s\right)$ field operations to compute
the determinant of a nonsingular matrix $\mathbf{F}\in\mathbb{K}\left[x\right]^{n\times n}$,
where $s$ is the average column degree of $\mathbf{F}$. \end{thm}
\begin{proof}
From \prettyref{lem:firstDiagonalBlock} and \prettyref{lem:secondDiagonalBlock}
the computation of the two diagonal blocks $\mathbf{G}_{1}$ and $\mathbf{G}_{2}$
costs $O^{\sim}\left(n^{\omega}s\right)$ field operations, which
dominates the cost of the other operations in the algorithm. 

Now for this recursive algorithm, if we let its cost on a subproblem
be $g(m)$ for the input matrix of dimension $m$, by \prettyref{lem:firstDiagonalBlock}
and \prettyref{lem:secondDiagonalBlock} the sum of the column degrees
of the input matrix is still bounded by $ns$, but the average column
is now $ns/m$. Then the cost 
\begin{eqnarray*}
g(m) & \in & O^{\sim}(m^{\omega}\left(ns/m\right))+g(\left\lceil m/2\right\rceil )+g(\left\lfloor m/2\right\rfloor )\\
 & \in & O^{\sim}(m^{\omega-1}ns)+2g(\left\lceil m/2\right\rceil )\\
 & \in & O^{\sim}(m^{\omega-1}ns).
\end{eqnarray*}
 The cost on the original problem when the dimension $m=n$ is therefore
$O^{\sim}\left(n^{\omega}s\right)$. %Note that always rounding up $n/2$ to $\left\lceil n/2\right\rceil $ is no worse than assuming 
%$n$ is a power of $2$. In other words, the entries in the sequence $\left[\left\lceil %n/2\right\rceil ,\left\lceil n/4\right\rceil ,\dots,1\right]$
%are not larger than the corresponding entries in the sequence $\left[m/2,m/4,\dots,1\right]$,
%where $m =2^{\left\lceil \log_{2}n\right\rceil }$ is the smallest power of $2$ that is not less %than $n$.
\end{proof}




\section{Conclusion\label{sec:Future-Research}}

In this paper we have given a new, deterministic algorithm for computing
the Hermite normal form of a nonsingular matrix polynomial. Our method
relied on the efficient, deterministic computation of the diagonal
elements of the Hermite form. As a corollary we also obtain a fast,
deterministic algorithm for finding the determinant of a matrix of
polynomials.

In terms of future research ... 


\bibliographystyle{plain}

%\bibliographystyle{plainnat}
\bibliographystyle{plain}
\bibliography{paper}

\end{document}
