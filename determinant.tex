
\subsection{Determinant of the unimodular matrix}

\prettyref{lem:firstDiagonalBlock} and \prettyref{lem:secondDiagonalBlock}
show that the two diagonal blocks in \prettyref{eq:step1HermiteDiagonal}
can be computed efficiently. To compute the determinant of $\mathbf{F}$
based on \prettyref{eq:determinantFromDiagonalBlocks}, we still need
to know the determinant of the unimodular matrix $\mathbf{U}$ that
satisfies \prettyref{eq:UPartitioned}, or equivalently, we can also
find out the determinant of $\mathbf{V}=\mathbf{U}^{-1}$. Note the
computation of the diagonal blocks in $\mathbf{G}$ also gives $\mathbf{U}_{R}$,
the right $n-k$ columns of $\mathbf{U}$, which is also a right kernel
basis of $\mathbf{F}_{U}$. In fact, it also gives us the matrix $\mathbf{V}_{U}$
consisting of the top $k$ rows of $\mathbf{V}$, from computing $\mathbf{G}_{1}$,
a column basis of $\mathbf{F}_{U}$.
\begin{lem}
Let $k$ be the dimension of $\mathbf{G}_{1}$. The matrix $\mathbf{V}_{U}\in\mathbb{K}\left[x\right]^{k\times n}$
satisfies $\mathbf{G}_{1}\mathbf{V}_{U}=\mathbf{F}_{U}$ if and only
if $\mathbf{V}_{U}$ is the submatrix of $\mathbf{V}=\mathbf{U}^{-1}$
consisting of the top $k$ rows of $\mathbf{V}$. \end{lem}
\begin{proof}
This can be seen from 
\[
\mathbf{G}\mathbf{V}=\begin{bmatrix}\mathbf{G}_{1} & 0\\
* & \mathbf{G}_{2}
\end{bmatrix}\begin{bmatrix}\mathbf{V}_{U}\\
\mathbf{V}_{D}
\end{bmatrix}=\begin{bmatrix}\mathbf{F}_{U}\\
\mathbf{F}_{D}
\end{bmatrix}=\mathbf{F}.
\]

\end{proof}
While the determinant of $\mathbf{V}$ or the determinant of $\mathbf{U}$
is needed to compute the determinant of $\mathbf{F}$, a major problem
is that we do not know $\mathbf{U}_{L}$ or $\mathbf{V}_{D}$, which
may not be efficiently computed due to their large degrees. This means
we need to compute the determinant of $\mathbf{V}$ or $\mathbf{U}$
without knowing the complete matrix $\mathbf{V}$ or $\mathbf{U}$.
The following lemma shows how this can be done using just $\mathbf{U}_{R}$
and $\mathbf{V}_{U}$.
\begin{lem}
\label{lem:scalingToDeterminant} Let $\mathbf{U}=\left[\mathbf{U}_{L},\mathbf{U}_{R}\right]$
and $\mathbf{F}$ be as before, that is, they satisfy 
\[
\mathbf{F}\mathbf{U}=\begin{bmatrix}\mathbf{F}_{U}\\
\mathbf{F}_{D}
\end{bmatrix}\begin{bmatrix}\mathbf{U}_{L} & \mathbf{U}_{R}\end{bmatrix}=\begin{bmatrix}\mathbf{G}_{1} & 0\\
* & \mathbf{G}_{2}
\end{bmatrix}=\mathbf{G},
\]
 where the row dimension of $\mathbf{F}_{U}$, the column dimension
of $\mathbf{U}_{L}$, and the dimension of $\mathbf{G}_{1}$ are $k$.
Let $\mathbf{V}=\begin{bmatrix}\mathbf{V}_{U}\\
\mathbf{V}_{D}
\end{bmatrix}$ be the inverse of $\mathbf{U}$ with $k$ rows in $\mathbf{V}_{U}$.
If $\mathbf{U}_{L}^{*}\in\mathbb{K}\left[x\right]^{n\times k}$ is
a matrix such that $\mathbf{U}^{*}=\left[\mathbf{U}_{L}^{*},\mathbf{U}_{R}\right]$
is unimodular, then 
\[
\det\mathbf{F}~=~\frac{\det\mathbf{G}\cdot\det\left(\mathbf{V}_{U}\mathbf{U}_{L}^{*}\right)}{\det\left(\mathbf{U}^{*}\right)}.
\]
\end{lem}
\begin{proof}
Since $\det\mathbf{F}~=~\det\mathbf{G}\cdot\det\mathbf{V}$, we just
need to show that $\det\mathbf{V}=\det\left(\mathbf{V}_{U}\mathbf{U}_{L}^{*}\right)/\det\left(\mathbf{U}^{*}\right)$,
which follows from 
\begin{eqnarray*}
\det\mathbf{V}\cdot\det\mathbf{U}^{*} & = & \det\left(\mathbf{V}\cdot\mathbf{U}^{*}\right)\\
 & = & \det\left(\begin{bmatrix}\mathbf{V}_{U}\\
\mathbf{V}_{D}
\end{bmatrix}\begin{bmatrix}\mathbf{U}_{L}^{*} & \mathbf{U}_{R}\end{bmatrix}\right)\\
 & = & \det\left(\begin{bmatrix}\mathbf{V}_{U}\mathbf{U}_{L}^{*} & 0\\
* & I
\end{bmatrix}\right)\\
 & = & \det\left(\mathbf{V}_{U}\mathbf{U}_{L}^{*}\right).
\end{eqnarray*}

\end{proof}
\prettyref{lem:scalingToDeterminant} shows that the determinant of
$\mathbf{V}$ can be computed using $\mathbf{V}_{U}$ and $\mathbf{U}_{R}$,
which we already know from the computation of the diagonal blocks
in $\mathbf{G}$, and a unimodular completion of $\mathbf{U}_{R}$.
In fact, this can be made more efficient still by noticing that the
higher degree parts do not affect the computation.
\begin{lem}
\label{lem:determinantOfUnimodular}If $\mathbf{U}\in\mathbb{K}\left[x\right]^{n\times n}$
is unimodular, then $\det\mathbf{U}=\det\left(\mathbf{U}\mod x\right)=\det\left(\mathbf{U}\left(0\right)\right)$.\end{lem}
\begin{proof}
Note that $\det\left(\mathbf{U}\left(\alpha\right)\right)=\left(\det\mathbf{U}\right)\left(\alpha\right)$
for any $\alpha\in\mathbb{K}$, that is, the result is the same whether
we do evaluation before or after computing the determinant. Taking
$\alpha=0$, then we have 
\[
\det\left(\mathbf{U}\mod x\right)=\det\left(\mathbf{U}\left(0\right)\right)=\left(\det\mathbf{U}\right)\left(0\right)=\det\left(\mathbf{U}\right)\mod x=\det\mathbf{U}.
\]

\end{proof}
This allows us to use just the degree zero coefficient matrices in
the computation. So \prettyref{lem:scalingToDeterminant} can be improved
as follows.
\begin{lem}
\label{lem:scalingToDeterminantSimplified} Let $\mathbf{F}$, $\mathbf{U}=\left[\mathbf{U}_{L},\mathbf{U}_{R}\right]$,
and $\mathbf{V}=\begin{bmatrix}\mathbf{V}_{U}\\
\mathbf{V}_{D}
\end{bmatrix}$ be as before. Let $U_{R}=\mathbf{U}_{R}\mod x$, $V_{U}=\mathbf{V}_{U}\mod x$
be the constant matrices of $\mathbf{U}_{R}$ and $\mathbf{V}_{U}$,
respectively. If $U_{L}^{*}\in\mathbb{K}^{n\times*}$ is a matrix
such that $U^{*}=\left[U_{L}^{*},U_{R}\right]$ is unimodular, then
\[
\det\mathbf{F}~=~\frac{\det\mathbf{G}\cdot\det\left(V_{U}U_{L}^{*}\right)}{\det\left(U^{*}\right)}.
\]
 \end{lem}
\begin{proof}
\prettyref{lem:determinantOfUnimodular} implies that $\det\mathbf{V}=\det V$
and $\det\mathbf{U}^{*}=\det U^{*}$, which can be then substituted
in the proof of \prettyref{lem:scalingToDeterminant} to obtain the
result.\end{proof}
\begin{rem}
\prettyref{lem:scalingToDeterminantSimplified} requires us to compute
$U_{L}^{*}\in\mathbb{K}^{n\times*}$ a matrix such that $U^{*}=\left[U_{L}^{*},U_{R}\right]$
is unimodular. This can be obtained easily from the unimodular matrix
that transforms $V_{U}$ to its reduced column echelon form computed
using the Gauss Jordan transform algorithm from \citep{storjohann:phd2000}
with a cost of $O\left(nm^{\omega-1}\right)$. 
\end{rem}
We now have all the ingredients needed for computing the determinant
of $\mathbf{F}$. A recursive algorithm is given in \prettyref{alg:determinant},
which computes the determinant of $\mathbf{F}$ as the product of
the determinant of $\mathbf{V}$ and the determinant of $\mathbf{G}$.
The determinant of $\mathbf{G}$ is computed by recursively computing
the determinants of its diagonal blocks $\mathbf{G}_{1}$ and $\mathbf{G}_{2}$.

\begin{algorithm}[t]
\caption{$\determinant(\mathbf{F})$}
\label{alg:determinant}

\begin{algorithmic}[1]
\REQUIRE{$\mathbf{F}\in\mathbb{K}\left[x\right]^{n\times n}$ is nonsingular.
}

\ENSURE{the determinant of $\mathbf{F}$.}



\STATE{\textbf{if }$n=1$ \textbf{then} \textbf{return} $\mathbf{F}$; \textbf{endif};}

\STATE{$\begin{bmatrix}\mathbf{F}_{U}\\
\mathbf{F}_{D}
\end{bmatrix}:=\mathbf{F}$, with $\mathbf{F}_{U}$ consists of the top $\left\lceil n/2\right\rceil $
rows of $\mathbf{F}$;}

\STATE{$\mathbf{G}_{1},\mathbf{U}_{R},\mathbf{V}_{U}:=\colBasis(\mathbf{F}_{U})$;
(Here we make $\colBasis()$ also return the kernel basis and the
right factor it computed.)}





\STATE{$\mathbf{G}_{2}:=\mathbf{F}_{D}\mathbf{U}_{R}$;}

\STATE{$U_{R}:=\mathbf{U}_{R}\mod x$; $V_{U}:=\mathbf{V}_{U}\mod x$;}

\STATE{compute $U_{L}^{*}\in\mathbb{K}^{n\times k}$ , a matrix that makes
the matrix $U^{*}=\left[U_{L}^{*},U_{R}\right]$ unimodular;}

\STATE{$d_{V}:=\det\left(V_{U}U_{L}^{*}\right)/\det(U^{*};$}

\STATE{$\mathbf{d}_{G}:=\determinant(\mathbf{G}_{1})\cdot\determinant(\mathbf{G}_{2});$}

\STATE{\textbf{return} $d_{V}\cdot\mathbf{d}_{1}\cdot\mathbf{d}_{2}$;}
\end{algorithmic}
\end{algorithm}


\begin{example}
Let 
\[
\mathbf{F}=\left[\begin{array}{rcccc}
x & -{x}^{3} & -2\,{x}^{4} & 2x & -{x}^{2}\\
\noalign{\medskip}1 & -1 & -2\, x & 2 & -x\\
\noalign{\medskip}-3 & 3\,{x}^{2}+x & 2\,{x}^{2} & -\,{x}^{4}+1 & 3\, x\\
\noalign{\medskip}0 & 1 & {x}^{2}+2\, x-2 & \,{x}^{3}+2x-2 & 0\\
\noalign{\medskip}1 & -{x}^{2}+2 & -2\,{x}^{3}-3\, x+3 & 2x+2 & 0
\end{array}\right]
\]
 working over $\mathbb{Z}_{7}[x]$. If $\mathbf{F}_{U}$ denotes the
top three rows of $\mathbf{F}$ then a column basis 
\[
\mathbf{G}_{1}=\left[\begin{array}{rcr}
x & -{x}^{3} & -2\,{x}^{4}\\
\noalign{\medskip}1 & -1 & -2\, x\\
\noalign{\medskip}-3 & 3\,{x}^{2}+x & 2\,{x}^{2}
\end{array}\right]
\]
 and kernel basis 
\[
\mathbf{U}_{R}=\left[\begin{array}{rc}
-1 & x\\
-x^{2} & 0\\
-3x & 0\\
-3 & 0\\
0 & 1
\end{array}\right]
\]
 were given in Example \ref{ex:example1}. The computation of the
column basis also gives the right factor 
\[
\mathbf{V}_{U}=\left[\begin{array}{cccrr}
1 & 0 & 0 & 2 & -x\\
0 & 1 & 0 & 2x^{2} & 0\\
0 & 0 & 1 & -x & 0
\end{array}\right].
\]
Then 
\[
{U}_{R}=\left[\begin{array}{rc}
-1 & 0\\
0 & 0\\
0 & 0\\
-3 & 0\\
0 & 1
\end{array}\right]~\mbox{and}~~{V}_{U}=\left[\begin{array}{cccrr}
1 & 0 & 0 & 2 & 0\\
0 & 1 & 0 & 0 & 0\\
0 & 0 & 1 & 0 & 0
\end{array}\right].
\]
A unimodular completion of $U_{R}$ is given by 
\[
U_{L}^{*}=\left[\begin{array}{ccc}
1 & 0 & 0\\
0 & 1 & 0\\
0 & 0 & 1\\
0 & 0 & 0\\
0 & 0 & 0
\end{array}\right].
\]
 The determinant of $\mathbf{V}$ can then be computed as 
\[
\frac{\det\left(V_{U}U_{L}^{*}\right)}{\det\left(U^{*}\right)}=-\frac{1}{3}=2.
\]
From recursively computing the determinant of $\mathbf{G}_{1}$ and
the determinant of $\mathbf{G}_{2}$, we obtain $\det\mathbf{G}_{1}=x^{6}-x^{4}$
and $\det\mathbf{G}_{2}=x^{4}-x$. 

Then the determinant of $\mathbf{V}$ can be computed as
\[
\det\mathbf{V}=\det\mathbf{V}\cdot\det\mathbf{G}_{1}\cdot\det\mathbf{G}_{2}=2\left(x^{6}-x^{4}\right)\left(x^{4}-x\right)=2x^{10}-2x^{8}-2x^{7}+2x^{5}.
\]
\qed \end{example}

