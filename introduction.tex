
\section{Introduction}

For a given square, nonsingular matrix polynomial $\mathbf{F}\in\mathbb{K}[x]^{n\times n}$
there exists a unimodular matrix $\mathbf{U}$ such that $\mathbf{F}\cdot\mathbf{U}=\mathbf{H}$,
a matrix in (column) Hermite normal form. Thus 
\[
\mathbf{H}=\left[\begin{array}{cccc}
h_{11}\\
h_{21} & h_{22}\\
\vdots & \vdots & \ddots\\
h_{n1} & \cdots & \cdots & h_{nn}
\end{array}\right]
\]
 a lower triangular matrix where $h_{ii}$ is monic and deg $h_{ij}<\mbox{ deg }h_{ii}$
for all $j<i$. Other variations include specifying row rather than
column forms (in which case the unimodular matrix multiplies on the
left rather than the right) or upper rather than lower triangular
form. The Hermite form was first defined by Hermite in 1851 in the
context of triangularizing integer matrices.

There has been considerable work on fast algorithms for Hermite form
computation. This includes $O^{\sim}\left(n^{4}d\right)$ algorithms
from Hafner and McCurley \citet{hafner} and Iliopoulos \citet{iliopoulos}
which control intermediate size by working modulo the determinant.
Hafner and McCurley \citet{hafner}, Storjohann and Labahn \citet{storjohann-labahn96}
and Villard \citet{villard96} gave new algorithms which reduced the
cost to $O^{\sim}\left(n^{\omega+1}d\right)$ operations, with $2<\omega<3$
being the exponent of matrix multiplication. The second named worked
with integer matrices but the results carried over directly to polynomial
matrices. Mulders and Storjohann \citet{mulders-storjohann:2003}
gave an iterative algorithm having complexity $O\left(n^{3}d^{2}\right)$,
thus reducing the exponent of $n$ at the cost of increasing the exponent
of the degree $d$.

During the past decade the goal has been to give an algorithm that
computes the Hermite form in the time it takes to multiply two polynomial
matrices having the same size $n$ and degree $d$ as the input matrix,
namely at a cost $O^{\sim}\left(n^{\omega}d\right)$. Such algorithms
already exist for a number of other polynomial matrix problems. This
included probabalistic algorithms for example linear solving \citet{mulders-storjohann:2003},
row reduction \citet{Giorgi2003} and polynomial matrix inversion
\citet{jeannerod-villard:05} and later deterministic algorithms for
linear solving and row reduction \citet{GSSV2012}. In the case of
Hermite normal form computation Gupta and Storjohann \citet{GS2011}
gave an algorithm which reduced the complexity to $O^{\sim}\left(n^{3}d\right)$.
Their algorithm is randomized of Las Vegas type rather than deterministic.

In this paper we give a deterministic Hermite norml form algorithm
having complexity $O^{\sim}\left(n^{3}d\right)$. As in Gupta and
Storjohann our approach is given in two steps : first find the diagonal
elements of the Hermite form and then find the remaining entries.
While Gupta and Storjohann use the Smith normal form to determine
the diagonal entries our approach is to make use of fast, deterministic
methods forshifted minimal kernel basis and column basis computation.
The use of shifted minimal kernel bases for matrix normal form computation
was previously used in \citet{BLV:1999,BLV:jsc06} in order to obtain
efficient algorithms in the case where intermediate coefficient growth
is a concern.

Our algorithm finds the diagonal entries of the Hermite form has a
cost of $O^{\sim}\left(n^{\omega-1}s\right)$ field operations where
$s$ is the average of the column degrees of $\mathbf{F}$. By slightly
modifying the diagonal algorithm we are also able to give as a corollary
a $O^{\sim}\left(n^{\omega-1}s\right)$ deterministic algorithm for
finding the determinant of $\mathbf{F}$. This compares to the $O^{\sim}\left(n^{\omega}d\right)$
cost for the Las Vegas determinant algorithm of Storjohann given in
\citet{storjohann:2002,storjohann:2003}

The remainder of this paper is organized as follows. In the next section
we give preliminary information for shifted degrees, kernel and column
bases of polynomial matrices. Section 3 then contains the algorithm
for finding the diagonal elements of a Hermite form with the following
section giving the details of the fast algorithm for the entire Hermite
normal form computation. Section 5 describes the small modification
needed for determinant computation. The paper ends with a conclusion
and topics for future research. 
